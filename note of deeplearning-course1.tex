\documentclass[a4paper,12pt]{article}

\usepackage{CJK,indentfirst}%indenfirst宏包允许设置首行缩进
\usepackage{amsmath}
\usepackage{graphicx}
\begin{document}

\begin{CJK*}{UTF8}{gbsn}%用宋体

\part{Neural Networks and Deep Learning}
 \section{week1:科普.what is deep learning?}
  \begin{flushleft}
1、基本概念:
(1)缩写\\
\qquad RELU:rectified linear unit\\
\qquad CNN:convolutional network\\
\qquad RNN:recurrent NN\\
(2)小知识点:\\
\qquad structured data:就是标准数据表格\\
\qquad unstructured Data:声音信号,图像,文本..\\
2、应用:\\
\quad CNN:online advertising、photo tagging\\
\quad RNN: speech recognition\\
\quad RNNS: machine translation\\
\quad HYBIRD: auto driving\\
3、深度学习的驱动因素\\
\quad  (1)scale drives deep learning progress(神经网络or训练集)\\
\quad (2)算法的进化;sigmoid RELU\\
\quad (3)faster computation\\

\end{flushleft}
\section{week2 神经网络的编程基础 \\basics of neural network programming}
 \subsection{01 二分类(binary classification)}
  \begin{flushleft}
    符号说明:\\
\quad    1.第i个样本x$_{(i)}$=[255,215...]:存储了rgb三个通道所有像素值:64*64*3(也即维度)\\
\quad    2.m表示训练样本数量,M$ _{test}$,M$_{train}$\\
\quad    3.X表示总的输入矩阵\\
\quad    4.Y.shape=(1,m)\\
\quad    5.n$_x$:input size 输入规模,n$_y$:output size 输出规模\\
   \end{flushleft}   
    \subsection{02 逻辑回归 Logistic Regression}
\begin{flushleft}
    1.yhat=P(y=1$|$ $_X$)\verb|-->|y hat 意思是对于x,输出y=1的几率\\
    2.yhat=sigmoid(W$^{T}$X+b)\\
\end{flushleft}   

    \subsection{03 逻辑回归的代价函数 cost function of logistic regression}
\begin{flushleft}   
    1、loss function:L(yhat,y)=-[ylogyhat+(1-y)log(1-yhat)]\\
    2、cost function: L(w,b)=1/m +$\sum$\L(yhat,y)\\
\end{flushleft} 
    \subsection{04 梯度下降 Gradient Descent}
\begin{flushleft}     
    1、J is a convex function\\
    2、几乎所有方法对于初始化J都有用,通常初始化wb为0\\
    3、求J的最小值就是利用求偏导数来确定下降的方向,并依靠α确定每一步下降的距离,并这样一步一步不断迭代直到得到的最小值.\\
\end{flushleft}  
    \subsection{11,12 向量化 Vectorization}
\begin{flushleft}

    
   举了两个向量相乘的栗子和普通for循环作对比,看出向量化特别的快\\
    BTW,常用对向量进行的操作\\   (1)np.dot(a,c)意为两个矩阵点乘\\
    					    (2)np.exp(v)意为将v的每项指数化并求和 \\
    					    (3)np.log表示求对数,np.abss表示求绝对值\\
    					    (4)np.maximum,v**2\\
                            (5)把向量拍扁:\\
                            flatten a matrix X of shape (a,b,c,d) to a matrix X\verb|_| flatten of shape    
                            (b∗∗c∗∗d, a) is to use:\\
                            X\verb|_|flatten = X.reshape(X.shape[0], -1).T     X.T is the transpose of X
\end{flushleft}
     \subsection{assignment:logistic regression with a neural net work mindset}
\begin{flushleft}     
      1、steps for pre-processing a new dataset are:

    (1)看清楚模样:Figure out the dimensions and shapes of the problem (m\verb|_|train, m\verb|_|test, num\verb|_|px, ...)\\
\qquad  多少个受训的:m\verb|_|train = train\verb|_|set\verb|_|x\verb|_|orig.shape[0]\\
\qquad  多少个小白鼠:m\verb|_|test = test\verb|_|set\verb|_|x\verb|_|orig.shape[0]\\
\qquad  每个边的像素:num\verb|_|px = train\verb|_|set\verb|_|x\verb|_|orig.shape[1]\\
    (2)拍扁:Reshape the datasets such that each example is now a vector of size (num\verb|_|px * num\verb|_|px * 3, 1)\\
        use:train\verb|_|set\verb|_|x\verb|_|flatten=train\verb|_|set\verb|_|x\verb|_|orig.reshape(train\verb|_|set\verb|_|x\verb|_|orig.shape[0],-1).T\\
    (3)九九归一:"Standardize" the data\\
    use:train\verb|_|set\verb|_|x = train\verb|_|set\verb|_|x\verb|_|flatten/255.\\
\end{flushleft}
\section{week3 神经网络的编程基础 basics of neural network programming}
  \subsection{01-Neural Networks Overview}
\begin{flushleft}
1、符号表示:上标方括号表示第几层,上标圆括号表示第几个样本\\
\end{flushleft}
  \subsection{02 neural network representation神经网络的表示}
  1、hidden layer之所以叫hidden是因为我们在训练是可以看到输入值和准确值,但hidden值是看不到的\\
  2、 a$^{[0]}$  用来表示输入层(即那一堆x),a$^{[1]}$表示hidden layer...\\
  且数层数的时候,往往不把输入层算作层数里面。\\
  \subsection{03 computin a Neural Network's output}
  1、其实就是将前面学过的regression在每个x上都来一次\\
  2、重点还是要理解矩阵思想 Z$^{[1]}$=W$^{[1]}$a$^{[0]}$+b$^{[1]}$\\
  \subsection{04 vectorizing across multiple examples}
\begin{flushleft}
1、for i=1 to m:\\
\qquad Z$^{[1](i)}$=W$^{[1]}$x$^{(i)}$+b$^{[1]}$\\
\qquad a$^{[1](i)}$=σ(Z$^{[1](i)}$)\\
\qquad Z$^{[2](i)}$=W$^{[2]}$a$^{[1](i)}$+b$^{[2]}$\\
\qquad a$^{[2](i)}$=σ(Z$^{[2](i)}$)\\
就是这样,然后把Z挨个排起来形成矩阵就好了,X也是是x的横向堆叠。
\end{flushleft}
\subsection{05 向量化实现解释Explanation for vectorized implementation}



\includegraphics[width=4.5in,height=3.5in]{vectorizing.jpg}
\subsection{06 Activation function}
\begin{flushleft}
1、除了输出层,tanh函数的表现几乎都要比sigmoid函数要好\\
2、ReLu:a=max(0,z),即z大于0导数为1,a=z否则a=0;\\
3、经验之谈:\\
\quad (1)如果输出为0or1,sigmoid很适合作为输出层的激活函数。\\
\quad (2)其他单元都用ReLu\\
\end{flushleft}
\includegraphics[width=4.5in,height=1.5in]{activation_function.jpg}\\
\subsection{07 Why do you need non-linear activation functions?}
\begin{flushleft}
1、如果激活函数是线性的,那就是一直拿着一条直线转圈,缺乏非线性表出的能力\\
\end{flushleft}
\subsection{08 Derivatives of activation functions}
\begin{flushleft}
1、sigmoid函数的导数为g(z)(1-g(z))\\
\includegraphics[width=4.9in,height=1.9in]{derivative_sigmoid.jpg}\\
2、tanh函数的导数为1-(tanh)$^2$\\
\includegraphics[width=4.9in,height=1.9in]{derivative_tanh.jpg}\\
3、ReLu的导数很简单,就是分段\\
\includegraphics[width=4.9in,height=1.9in]{derivative_ReLu.jpg}\\
\end{flushleft}
\subsection{09 Gradient descent for neural networks}
\begin{flushleft}
1、cost function :\\
\begin{equation*}
J = \frac{1}{m}\sum_{i=1}^nL(yhat,y)   
\end{equation*}
2、六个重要公式\\
\includegraphics[width=4.9in,height=1.9in]{computing_derivatives.jpg}\\
\end{flushleft}
\subsection{10 Backpropagation intuition}
\begin{flushleft}
1、回顾一下logistic regression:\\
1)\[ \frac{\partial L}{\partial a}
=\left. \frac{-y}{a}
+ \frac{1-y}{1-a}\right.\]
2) \[ \frac{\partial L}{\partial a}=a-y\]
3)
\begin{center}
dw=dz*x\\
db=dz\\
\end{center}
2、神经网络的后向传播和logistic regression差不多,尤其是从L经过a$^{[2]}$到z$^{[2]}$的过程几乎是一样的\\
比较特殊的一个dZ$^{[1]}$=W$^{[2]T}$dZ$^{[2]}$*g$^{[1]'}$(Z$^{[1]}$)\\
下图是neural network gradient的全部公式:\\
\includegraphics[width=4.9in,height=1.9in]{neural_gradients.jpg}\\
3.下图是对梯度下降公式的一个总结:\\
\includegraphics[width=4.9in,height=2.3in]{summary_gradient_descent.jpg}\\
\end{flushleft}
\subsection{11 Random Initialization}
1、不要都初始化为0(这样的化多层隐藏单元就没有意义了)而是应该选择随机初始化。\\
2、初始化语句:\\
\quad W$^{[1]T}$=np.random.randn((2,2))*0.01\\
\quad b$^{[1]T}$=np.zero((2,1)),因为b无所谓随不随机\\  
W$^{[2]T}$和bW$^{[2]T}$初始化与上面相同\\
W要乘0.01是因为如果数字太大的化,最终回落在sigmoid或tanh的平缓部分,这样会使得学习效率很低\\
\section{week4  Deep layer Neural network}
\subsection{01 Deep L-layer Neural network}
\begin{flushleft}
1、L表示神经网络的层数(不包括输入层),下图是具体的栗子:\\
\includegraphics[width=5.3in,height=2.8in]{DNN_notation.jpg}\\
\end{flushleft}
\subsection{02 forward and backward propagation}
\begin{flushleft}
1、Z$^{[l]}$=W$^{[l]}$A$^{[l-1]T}$+b$^{[l]}$ \\
2、A$^{[l]}$=g$^{[l]}$(Z$^{[l]}$)
3、反向传播的公式如下:\\
\includegraphics[width=5.3in,height=2.8in]{backpro.jpg}\\
4、四层nn的前向传播和后向传播流程图如下:\\
\includegraphics[width=5.3in,height=2.8in]{structure_pro_back.jpg}\\
\subsection{04 Getting your matrix dimensions right}
\begin{flushleft}
1、根据矩阵的乘法规则,我们知道,W$^{[l]}$的维度一定是(n$^{[l]}$,n$^{[l-1]}$)\\
2、同理,b$^{[l]}$的维度一定是(n$^{[l]}$,1)\\
3、当然dW的维度和W一样,db的维度和b一样\\
4、Z和a的维度应该相等,为(n$^{[l]}$,m),其中m是训练集的个数\\
5、老师笔记如下:\\
\includegraphics[width=5.3in,height=2.8in]{dimension_parameter.jpg}
\end{flushleft}
\subsection{05 Why deep representations?}
\begin{flushleft}
1、人脸识别的大概流程(整体思想就是从简单到复杂):第一层,先用边缘检测识别边缘,再识别一个部位,再逐步扩大\\
2、语音识别也是类似,先分辨声音高低是不是杂音,再识别音位再识别单词再识别句子\\
3、但也不要一味的用太多层数,适当就好\\
\end{flushleft}
\subsection{06 Building blocks of deep neural network}
\begin{flushleft}
1、以下是NN实现时的总体结构图:\\
\includegraphics[width=5.3in,height=2.8in]{for_back_fun.jpg}\\

\end{flushleft}
\end{flushleft}
\end{CJK*}
\end{document}